\documentclass[11 pt]{amsart}

\usepackage[margin=1.3in]{geometry}

\usepackage{amssymb}
\usepackage{amsmath}
\usepackage{amsthm}
\usepackage{amsfonts}
\usepackage{amsxtra}
\usepackage[all,2cell]{xy}
\usepackage{verbatim}
\usepackage{color}
\usepackage[unicode=true, pdfusetitle,
 bookmarks=true,bookmarksnumbered=false,
 breaklinks=false,
 backref=false,
 colorlinks=true,
 linkcolor=blue,
 citecolor=blue,
 urlcolor=blue,
 final
]{hyperref}

\usepackage{todonotes}
\usepackage{bookmark}
\usepackage{ytableau}

\usepackage[capitalise]{cleveref}
\newcommand{\fref}{\cref}
\newcommand{\Fref}{\Cref}
\newcommand{\prettyref}{\cref}
\newcommand{\newrefformat}[2]{}



\theoremstyle{plain}   % This is the default, anyway
\newtheorem{thm}{Theorem}[section] % numbered theorem
\makeatletter\let\c@thm\c@thm\makeatother
\newtheorem{cor}{Corollary}[section]
\makeatletter\let\c@cor\c@thm\makeatother
\newtheorem{lemma}{Lemma}[section]
\makeatletter\let\c@lemma\c@thm\makeatother
\newtheorem{prop}{Proposition}[section]
\makeatletter\let\c@prop\c@thm\makeatother
\newtheorem{claim}{Claim}[section]
\makeatletter\let\c@claim\c@thm\makeatother



\newtheorem*{unnumberedtheorem}{Theorem}  % unnumbered theorem
\newtheorem*{unnumberedcorollary}{Corollary}
\newtheorem*{unnumberedlemma}{Lemma}
\newtheorem*{unnumberedproposition}{Proposition}
\newtheorem*{unnumberedclaim}{Claim}

\theoremstyle{definition}

\newtheorem{defn}{Definition}[section]
\makeatletter\let\c@defn\c@thm\makeatother
\newtheorem{const}{Construction}[section]
\makeatletter\let\c@const\c@thm\makeatother
\newtheorem{notn}{Notation}[section]
\makeatletter\let\c@notn\c@thm\makeatother
\newtheorem{outline}{Proof Outline}[section]
\makeatletter\let\c@outline\c@thm\makeatother


\theoremstyle{remark}

\newtheorem{rem}{Remark}[section]
\makeatletter\let\c@rem\c@thm\makeatother
\newtheorem{ex}{Example}[section]
\makeatletter\let\c@ex\c@thm\makeatother
\newtheorem{observation}{Observation}[section]
\makeatletter\let\c@observationn\c@thm\makeatother

\newtheorem*{unnumberedtheoremremark}{Remark}
\newtheorem*{unnumberedtheoremexample}{Example}
\newtheorem*{unnumberedtheoremdefinition}{Defintion}

\makeatletter
\let\c@equation\c@thm
\numberwithin{equation}{section}
\makeatother

\def\cA{\mathcal{A}}\def\cB{\mathcal{B}}\def\cC{\mathcal{C}}\def\cD{\mathcal{D}}\def\cE{\mathcal{E}}\def\cF{\mathcal{F}}\def\cG{\mathcal{G}}\def\cH{\mathcal{H}}\def\cI{\mathcal{I}}\def\cJ{\mathcal{J}}\def\cK{\mathcal{K}}\def\cL{\mathcal{L}}\def\cM{\mathcal{M}}\def\cN{\mathcal{N}}\def\cO{\mathcal{O}}\def\cP{\mathcal{P}}\def\cQ{\mathcal{Q}}\def\cR{\mathcal{R}}\def\cS{\mathcal{S}}\def\cT{\mathcal{T}}\def\cU{\mathcal{U}}\def\cV{\mathcal{V}}\def\cW{\mathcal{W}}\def\cX{\mathcal{X}}\def\cY{\mathcal{Y}}\def\cZ{\mathcal{Z}}

\def\AA{\mathbb{A}} \def\BB{\mathbb{B}} \def\CC{\mathbb{C}} \def\DD{\mathbb{D}} \def\EE{\mathbb{E}} \def\FF{\mathbb{F}} \def\GG{\mathbb{G}} \def\HH{\mathbb{H}} \def\II{\mathbb{I}} \def\JJ{\mathbb{J}} \def\KK{\mathbb{K}} \def\LL{\mathbb{L}} \def\MM{\mathbb{M}} \def\NN{\mathbb{N}} \def\OO{\mathbb{O}} \def\PP{\mathbb{P}} \def\QQ{\mathbb{Q}} \def\RR{\mathbb{R}} \def\SS{\mathbb{S}} \def\TT{\mathbb{T}} \def\UU{\mathbb{U}} \def\VV{\mathbb{V}} \def\WW{\mathbb{W}} \def\XX{\mathbb{X}} \def\YY{\mathbb{Y}} \def\ZZ{\mathbb{Z}}  

\def\fa{\mathfrak{a}} \def\fb{\mathfrak{b}} \def\fc{\mathfrak{c}} \def\fd{\mathfrak{d}} \def\fe{\mathfrak{e}} \def\ff{\mathfrak{f}} \def\fg{\mathfrak{g}} \def\fh{\mathfrak{h}} \def\fj{\mathfrak{j}} \def\fk{\mathfrak{k}} \def\fl{\mathfrak{l}} \def\fm{\mathfrak{m}} \def\fn{\mathfrak{n}} \def\fo{\mathfrak{o}} \def\fp{\mathfrak{p}} \def\fq{\mathfrak{q}} \def\fr{\mathfrak{r}} \def\fs{\mathfrak{s}} \def\ft{\mathfrak{t}} \def\fu{\mathfrak{u}} \def\fv{\mathfrak{v}} \def\fw{\mathfrak{w}} \def\fx{\mathfrak{x}} \def\fy{\mathfrak{y}} \def\fz{\mathfrak{z}}
\def\fN{\mathfrak{N}}
\def\fgl{\mathfrak{gl}}  \def\fsl{\mathfrak{sl}}  \def\fso{\mathfrak{so}}  \def\fsp{\mathfrak{sp}}  
\def\GL{\mathrm{GL}} \def\SL{\mathrm{SL}}  \def\SP{\mathrm{SL}}\def\OG{\mathrm{O}}

\def\aa{\mathbf{a}} \def\bb{\mathbf{b}} \def\cc{\mathbf{c}} \def\dd{\mathbf{d}} \def\ee{\mathbf{e}} \def\ff{\mathbf{f}}
%\def\gg{\mathbf{g}}
\def\hh{\mathbf{h}} \def\ii{\mathbf{i}} \def\jj{\mathbf{j}} \def\kk{\mathbf{k}}
%\def\ll{\mathbf{l}}
\def\mm{\mathbf{m}} \def\nn{\mathbf{n}} \def\oo{\mathbf{o}} \def\pp{\mathbf{p}} \def\qq{\mathbf{q}} \def\rr{\mathbf{r}} \def\ss{\mathbf{s}}  \def\uu{\mathbf{u}} \def\vv{\mathbf{v}} \def\ww{\mathbf{w}} \def\xx{\mathbf{x}} \def\yy{\mathbf{y}} \def\zz{\mathbf{z}}
\def\zzero{\mathbf{0}}

\def\<{\langle} \def\>{\rangle}
\def\Aut{\mathrm{Aut}}
\def\ch{ \stackrel}
\def\col{\mathrm{col}}
\def\dim{\mathrm{dim}} 
\def\End{\mathrm{End}} 
\def\ev{\mathrm{ev}} 
\def\f{\varphi}
\def\gcd{\mathrm{gcd}}
\def\half{\hbox{$\frac12$}}
\def\Hom{\mathrm{Hom}}
\def\img{\mathrm{img}}
\def\id{\mathrm{id}} 
\def\Inn{\mathrm{Inn}}
\def\lcm{\mathrm{lcm}}
\def\normeq{\trianglelefteq}
%\def\ch{ \stackrel{\mathrm{ch}}{\trianglelefteq}}
%\def\normeq{\trianglelefteq}
\def\nul{\mathrm{nullity}}
\def\row{\mathrm{row}}
\def\rk{\mathrm{rank}}
\def\sgn{\mathrm{sgn}}
\def\sp{\mathrm{span}}
\def\supp{\mathrm{supp}}
\def\Syl{\mathrm{Syl}}
\def\tr{\mathrm{tr}} 
\def\vep{\varepsilon}

%\usepackage{mathabx}
\def\acts{\circlearrowright} %group action

 %Cleveref definitions
\crefname{lemma}{Lemma}{Lemmas}
\crefname{thm}{Theorem}{Theorems}
\crefname{defn}{Definition}{Definitions}
\crefname{notn}{Notation}{Notations}
\crefname{const}{Construction}{Constructions}
\crefname{prop}{Proposition}{Propositions}
\crefname{rem}{Remark}{Remarks}
\crefname{cor}{Corollary}{Corollaries}
\crefname{equation}{Diagram}{Diagrams}
\crefname{ex}{Example}{Examples}







\title{Representations of $S_n$ and their correspondence with integer partitions of $n$}

\author{Dale D. Schandelmeier-Lynch}






\pagestyle{plain}

\parindent0pt
\parskip4pt



\begin{document}



\begin{abstract}
  First we introduce some fundemental representation theory definitions and theorems neccesary to understand our results.
  Next the bijection between irreducible representations of $S_n$ and integer partitions of $n$ is established.
  Tabloids, their corresponding permutations, and Young symetrizers are used to define the $\CC[S_n]$-modules $M^\lambda$ and $S^\lambda$.
  We then prove that the set of Specht modules $S^\lambda$ over all partitions $\lambda$ of $n$ form a complete set of irreducible modules/representations
  of $S_n$.
  Finally we prove that elements in bijection with the standard Young tableaux of a given partition $\lambda$ form a basis for the Specht modules
  $S^\lambda$.
 
\end{abstract}


\maketitle

\section{Background}

In order to understand the relationship between Young tableaux and representations,
we'll first need a crash course in the basics of representation theory.
Fundemental abstract algebra knowledge will be assumed --- examples of concepts used include group actions, conjugacy classes, and
permutation facts.
Our first definition will be specific to the set of invertible $d \times d$ matrices with entries in $\CC$.
This is the full complex matrix algebra, because we have a vector space of matrices scaled by the the complex numbers
with a multiplication defined by matrix multiplication.
\begin{defn}[{\cite[Definition 1.2.1]{sagan}}]
  A \emph{representation} (more specifically a \emph{matrix representation}) of a group $G$ is a group homomorphism
  \[\rho: G \to \GL_d\]
  where $\GL_d$ is the complex general linear group of degree $d$; that is, the set of
  invertible $d \times d$ matrices from $\CC^d$ to $\CC^d$. \\
  To break down the definition of a group homomorphism, this means that
  \[ \rho(e) = I_d\]
  where $e$ is the identity of the group, and $I_d$ is the identity matrix of degree $d$ in $\CC$, and
  \[\rho(g)\rho(h) = \rho(gh) \text{ for all $g,h \in G$.}\]
    This means that multiplying group elements and then taking their matrix representation is equivalent to
    taking their matrix representations and performing matrix multiplication on them.
  \end{defn}
  Note that representations can be defined for other fields like $\RR$ or $\ZZ/P\ZZ$, and doing so changes the core theorems of the theory.
  A point of confusion with representations is that they are defined to be \emph{homomorphisms}, not \emph{monomorphisms}.
  We might inuitively imagine that a matrix representation is equivalent to the group just with matrices instead of group elements,
  but information/group elements can be lost by mapping nontrivial elements of the group to the identity matrix in the linear group
  (that is, there can be nontrivial kernel.)
  Furthermore, additional representations can be created by choosing a different degree or performing a change of basis on a given representation.
  A small degree can possibly restrict the possible representations to exclude monomorphisms.
  On the other hand, a change of basis shouldn't fundementally change the linear transformation a matrix represents.
  We can consider representations $\rho, \phi$ equivalent (through an equivalence relation) if there exists an invertible matrix $T$
  such that $\rho(g) = T^{-1} \phi(g) T$ for all $g \in G$.\par
  This relationship between matrices and linear transformations can be further used to avoid a choice of basis; we now introduce
  $G$-modules.
  \begin{defn}[{\cite[Definition 1.3.1]{sagan}}]
    Let $V$ be a vector space and $G$ be a group. Then $V$ is a \emph{$G$-module} if there is a group homomorphism
    \[\rho: G \to \GL(V),\]
    where $\GL(V)$ is the set of general linear transformations of the vector space $V$.
    Equivalently, V is a $G$-module if there is a multiplication $g \cdot v$ such that
    \begin{enumerate}
    \item $g \cdot v \in V$,
    \item $g(cv + dw) = c(gv) + d(gw)$,
    \item $(gh)v = g(hv)$, and
    \item $ev = v$
    \end{enumerate}
    for all $g,h \in G$, $v,w \in V$, and scalars $c,d \in \CC$.
    That is, there is a left action $G \acts V$ on $V$ that distributes over addition.
  \end{defn}
  These two definitions are equivalent because acting by a group element $g$ is equivalent to applying the linear transformation $\rho(g)$.
  $G$-modules and matrix representations are closely related. We use any matrix representation $X$ with degree matching the dimension of $V$
  to construct a $G$-module by defining the linear action $g \cdot v := X(g) v$. Likewise, if we have a $G$-module we can define
  a matrix representation $X$ by choosing a basis $B$ for the linear transformations $\rho(g)$ and then construct $X(g)$.
  An important difference between representations and modules is that we have a choice of vector space when making a module.
  We can use this fact (and the fact that actions are part of our module definition) to naturally represent arbitary group actions.
  \begin{ex} We turn a group action $G \acts X$ into a $G$-module by first making the vectors of our space be
    formal linear sum of elements of the set: $[x] = \sum_{x_i \in X} \lambda_i x_i$.
    The group action then can then be linearly extended to act on $\CC[X]$, the vector space generated by $X$ over $\CC$:
    \begin{align*}
      G &\acts \CC [X] \\
      (g,[x]) &\mapsto \sum_{x_i \in X} \lambda_i (g \cdot x_i)
    \end{align*}
    Now we can define a $G$-module
    \[ \rho: G \to \GL(|X|, \CC [X])\]
    where $\rho(g)$ is the linear transformation induced by the action of $G$ on $X$; if we take the standard basis of $\CC [X]$ to be
    the set $X$, then $\rho(g)$ can be constructed as a matrix which permutes each $x_i$ according to the image of the action of $g$ on each $x_i$.
  \end{ex}
  This paper will focus on finding \emph{irreducible} representations/modules,
  which are fundemental to understanding the representation theory of a group because
  for representations over $\CC$ and similarly nice fields any representation can be represented as a product of irreducible representations.
  We will concern ourselves with irreducible modules primarily because they are easier to work with and are in close correspondence
  with irreducible representations. To do so we first need submodules.
  \begin{defn}[{\cite[Definition 1.4.1]{sagan}}]
    Let $V$ be a $G$-module. a \emph{submodule}
    of $V$ is a subspace $W$ that is closed under the action of $G$; that is,
    \[w \in W \implies gw \in W \text{ for all } g \in G.\]
  \end{defn}
  \begin{ex}
    Any $G$-module $V$ has the submodules $W=V$ as well as $W= \{0\}$, where $0$ is the zero vector.
    These two submodules are called trivial, and any other submodules are called nontrivial.
  \end{ex}
  \begin{defn}[{\cite[Definition 1.4.1]{sagan}}]
    A nonzero $G$-module $V$ is \emph{reducible} if it contains a non-trivial submodule $W$.
    Otherwise $V$ is said to be reducible.
  \end{defn}

  We will need the following definition later on to demonstrate that we have found a complete set of irreducible non-isomorphic $G$-modules.
  
  \begin{defn}[{\cite[Definition 1.6.1]{sagan}}]
    Let $V$ and $W$ be $G$-modules. Then a \emph{$G$-homomorphism} is a linear transformation $\Theta: V \to W$ such that
    \[\Theta(g \cdot v) = g \cdot \Theta(v)\]
    for all $g \in G$ and $v \in V$.\\
    A \emph{$G$-isomorphism} is a bijective $G$-homomorphism.
  \end{defn}
  
  Finally, we introduce two results without proof that we will need to analyze representations of $S_n$. The first one, called
  Matchske's theorem, is the theorem that tells us that any representation can be constructed out of irreducible representations.
  To understand it we first need the notion of a direct sum.
  \begin{defn}[{\cite[Definition 1.5.1]{sagan}}]
    Let $V$ be a vector space with subspaces $U$ and $W$.
    Then $V$ is the \emph{(internal) direct sum of $U$ and $W$}, written $V = U \oplus W$, if every $v \in V$ can be
    written uniquely as a sum
    \[v= u +w, \quad u \in U, w \in W\]
  \end{defn}
  \begin{thm}[{\cite[Theorem 1.5.3]{sagan}; Matchske's theorem}]
      Let $G$ be a finite group and let $V$ be a nonzero $G$-module. Then
      \[ V = W_1 \oplus W_2 \oplus \dots \oplus W_k\]
      where each $W_i$ is an irreducible $G$-submodule of $V$.
    \end{thm}
    
    This next proposition gives us results from character theory which characterize the relationship between irreducible representations
    and their corresponding group.
    
  \begin{prop}[{\cite[Proposition 1.10.1]{sagan}}]
      Let $G$ be a finite group and suppose that the set $V_i$ ranging over $i$ forms a complete list of
      distinct irreducible $G$-modules. Then
      \begin{enumerate}
      \item $\sum_i (\dim V_i)^2 = |G|$, and
      \item the number of $V_i$ equals the number of conjugacy classes of $G$.
      \end{enumerate}
    \end{prop}
  
  

    
% A short introduction of what the paper is about, including the main results. You can include here a bit of the theory behind the problem. It doesn't have to include any details, just the bare minimum needed to explain what is going on.

% The last paragraph should be an outline of the paper. Something like:

% In \cref{label} we explain how to use labels. \cref{citations} is devoted to the use of citations.

  \section{Results}\label{results}
  \subsection{Motivation}\label{motivation} \hfill\\
We will consider the symmetric group $S_n$ and its representations; in doing so, we will find how these representations correspond to Young tableaux. \par
We state without proof that the number of irreducible representations (up to isomorphism) of a group is equal to the number of conjugacy classes in
the group \cite[Proposition 1.10.1]{sagan}; this comes from character theory outside of our scope. Fundemental abstact algebra tells us that 
conjugacy classes of $S_n$ are in bijection with the cycle types of $S_n$,
and because cycle types are unordered lists we can map each cycle type to its unique weakly decreasing ordering.
This forms a bijection between cycle types of $S_n$ and integer partitions of $n$.
Composing bijections, we can conclude that each integer partition of $n$ uniquely corresponds to an irreducible representation of $S_n$! \par
Our goal will be to find these irreducible representations by creating their corresponding irreducible $S_n$-modules.
\subsection{Tableaux and Tabloids} \hfill\\
In order to use tableaux in $S_n$-modules, we'll need to define the action of $S_n$ on a tableaux of a partition of size $n$.
\begin{defn}
  A \emph{Young Tableaux} is a numbering of a Young diagram of size $n$ with the numbers $\{1 \cdots n\}$ with no
  repeats allowed. \par
  There is no ordering condition, unlike with standard Young tableaux (hereafter refered to as SYT);
  we go out of our way to define a plain tableaux because the definition of a tableaux can be inconsistent across sources.
\end{defn}
The action of $\sigma \in S_N$ on a tableaux $T$ of size $n$ is to replace the box $i$ in $T$ with the box $\sigma(i)$ in $\sigma \cdot T$.
\begin{defn}[{\cite[Pg. 84]{fulton}}]
  The \emph{row group} of $T$, denoted $R(T)$, is the set of permutations which permute the entries of each row among themselves.
  If $\lambda = ( \lambda_1 \leq \lambda_2 \leq \cdots \leq \lambda_k < 0)$, then $R(T)$ is a product of symmetric groups
  $S_{\lambda_1} \times S_{\lambda_2} \times \cdots \times S_{\lambda_k}$.
  In fact, if row 1 of $T$ contains the numbers $\{ j_1 \cdots i_1\} \subseteq [n]$, row 2 contains $\{j_2 \cdots i_2 \}$, et cetra,
  then $R(T)$ is the product of the automorphisms each set --- $S_{\{ j_1 \cdots i_1\}} \times S_{\{ j_2 \cdots i_2\}} \times \cdots \times S_{\{ j_k \cdots i_k\}}$.\par
  We analagously define the column group of $T$, $C(T)$, to be the set of permutations which permute the entries of the columns among themselves.
  This is equivalent to the row group of the transpose of the tableaux.
\end{defn}
These subgroups of $S_n$ are compatible with the action of $S_n$ on $T$ in the following way:
\[ R(\sigma \cdot T) = \sigma \cdot R(T) \cdot \sigma^{-1} \text{ and } C(\sigma \cdot T) = \sigma \cdot C(T) \cdot \sigma^{-1}.\]
\begin{ex}
  For the tableaux below,
  $(143) \in R(T)$ but $(12) \notin R(T)$.
  Analagously $(143) \notin C(T)$ and $(12) \in C(T)$.
  \[
    \begin{ytableau}
    1 & 4 & 3 \\
    2 & 5 \\
    \end{ytableau}
  \]
  If we perform the action $(12)(34) \cdot T$ we now see that $\sigma (143) \sigma^{-1} = (12)(34) (143) (12)(34) = (243) \in R(\sigma \cdot T)$.
  Intuitively conjugation is swapping elements of the underlying set undergoing an automorphism.
  \[
    \begin{ytableau}
    2 & 3 & 4 \\
    1 & 5 \\
    \end{ytableau}
  \]
\end{ex}
We now define tabloids, which will be used in our construction of $S_n$-modules.
\begin{defn}[{\cite[Pg. 85]{fulton}}]
  A \emph{tabloid} is an equivalence class of tableaux where two tableaux of the same shape are equivalent if their rows contain the same
  values. The tabloid containing $T$ is denoted $\{T \}$. Two tableaux are in the same class ($\{ T \} = \{ T'\}$) when $T'  = \sigma \cdot T$ for
  some $\sigma \in R(T)$.
\end{defn}

\begin{ex}
  \ytableausetup{tabloids,centertableaux}
  
  Tabloids are notated as tableaux without vertical lines to convey that changing the ordering of values within a row doesn't change the tabloid.
  \[
    \begin{ytableau}
      1 & 4 & 3 \\
      2 & 5 \\
    \end{ytableau}    =
    \begin{ytableau}
      4 & 1 & 3 \\
      2 & 5 \\
    \end{ytableau}
    \neq
    \begin{ytableau}
      4 & 5 & 3 \\
      2 & 1 \\
    \end{ytableau}
  \]
\end{ex}

\subsection{The modules $M^\lambda$ and $S^\lambda$} \hfill\\
Before defining $M^\lambda$,  we must first linearly extend $S_n$ to the group ring $\CC[S_n]$, the
formal linear combinations of permutations with coefficents in $\CC$ --- $\sum x_\sigma \sigma$.
Multiplication in the group ring is determined by composition.
Note that we can restrict a $\CC[S_n]$-module to an $S_n$-module by restricting $\CC[S_n]$ to the set
\[\{1_\CC \sigma | \sigma \in S_n\}\]
which is isomorphic to $S_n$, so we still obtain our desired $S_n$-module and can construct any desired matrix representations of $S_n$.

\begin{defn}[{\cite[Pg. 86]{fulton}}]
  We let $M^\lambda$ denote the complex vector space with basis the set of tabloids $\{T\}$ for a given partition $\lambda$ size $n$.\par
  Since $S_n$ acts on the set of tabloids, it acts on $M^\lambda$, and we can linearly extend this to an action of $\CC[S_n]$ on $M^\lambda$;
  namely
  \[ (\sum x_\sigma \sigma) \cdot (\sum x_T \{T\}) = \sum \sum x_\sigma x_T \{ \sigma \cdot T \}.\]
  Therefore $M^\lambda$ is a $\CC[S_n]$-module. 
\end{defn}
We now need some special elements to construct our desired submodule of $M^\lambda$:
\begin{defn}[{\cite[Pg. 86]{fulton}}]
  Given a tableaux $T$, we define the element $ b_T \in \CC[S_n]$:
  \[b_T = \sum_{q \in C(T)} \sgn(q)q.\]
  This is a \emph{Young symmetrizer}; there are two others we have left undefined as they are used in the symmetric (column-wise)
  definition of tabloids outside of our scope.
\end{defn}

\begin{defn}[{\cite[Pg. 86]{fulton}}]
  For each tableaux (not tabloid!) $T$ of shape $\lambda$ there is an element $v_T \in M_\lambda$
  defined by the formula
  \[v_T = b_T \cdot \{T\} = \sum \sgn(q) \{q \cdot T\}.\]
  Note that changing $T$ might not change the tabloid $\{T\}$ but could change $b_T$ resulting in a different element in $M^\lambda$.
\end{defn}

We can now define the subspace $S^\lambda$:
\begin{defn}[{\cite[Pg. 87]{fulton}}]
  The \emph{Specht module}, denoted \emph{$S^\lambda$}, is the subspace of $M^\lambda$ spanned by the elements $v_T$, as $T$ varies over all tableaux of
  $\lambda$.
\end{defn}
To prove that $S^\lambda$ is actually a module, we need to show that it is closed under the action of $\CC[S_n]$; namely,
we will show that $\sigma \cdot v_T = v_{\sigma \cdot T}$ for all tableaux $T$ and $\sigma \in S_n$.
\begin{proof}
  Recall that $C(\sigma \cdot T) = \sigma \cdot C(T) \cdot \sigma^{-1}$; we first observe that
    \begin{align*}
    \sigma \cdot v_{T} &= \sigma \cdot b_T \cdot \{ T\} \\
                       &= \sigma \cdot \sum_{ q \in C(T)} \sgn(q) \{q \cdot T\} \\
                       &= \sum_{ q \in C(T)} \sgn(q) \{\sigma \cdot q \cdot T\}. \\
  \end{align*}
  On the other hand,
    \begin{align*}
    v_{\sigma \cdot T} &= b_{\sigma \cdot T} \{ \sigma \cdot T\} \\
                       &= \sum_{q \in C(\sigma \cdot T)} \sgn(q) \{q \cdot \sigma \cdot T\}\\
                       &= \sum_{q \in \sigma \cdot C(T) \cdot \sigma^{-1}} \sgn(q) \{q \cdot \sigma \cdot T\}\\
                       &= \sum_{q \in C(T)} \sgn(\sigma q \sigma^{-1}) \{\sigma q \sigma^{-1} \cdot \sigma \cdot T\}\\
                       &= \sum_{q \in C(T)} \sgn(\sigma q \sigma^{-1}) \{\sigma \cdot q \cdot T\}\\      
  \end{align*}
  We know from abstact alegebra that conjugation doesn't change the sign of a permutation, so these two expressions are equal.\\
  (Note that \cite{fulton} doesn't prove this and instead leaves it as an excercise to the reader.)
\end{proof}

Because $\sigma \cdot v_T = v_{\sigma \cdot T} \in S^\lambda$, $S^\lambda$ is closed under $S_n$. Furthermore $S^\lambda$ is closed under $\CC [S_n]$, as we can
now calculate
\[(\sum x_\sigma \sigma) \cdot v_T = \sum x_\sigma v_{\sigma \cdot T} \in S^\lambda.\]
Because $S^\lambda$ is a subspace of $M^\lambda$, we have thus proven that
\begin{thm}
  $S^\lambda$ is a $\CC[S_n]$ submodule of $M^\lambda$.
\end{thm}
In fact, we know that $S^\lambda = \CC[S_n] \cdot v_T$ for \emph{any given} tableaux $T$,
because we can create any desired element in $M^\lambda$ like so:
\[\sum_{\{T'\} \in A \subseteq \text{set of tabloids}} x_{T'} v_{T'} = (\sum x_{T'} \sigma_{T'}) \cdot v_T,\]
where $\sigma_{T'} \cdot v_T = v_{\sigma_{T'} \cdot T} = v_{T'}$.
\subsection{Irreducibility and completeness of the $S^\lambda$ modules} \hfill\\
We now need to show that the modules $S^\lambda$ have our desired properties: every $S^\lambda$ is irreducible, no two $S^\lambda$ are isomorphic,
and any irreducible representation is isomorphic to some $S^\lambda$. Once we do this we will have shown that the set of modules $S^\lambda$
over partitions of a given size $n$
are in bijection with the irreducible representations of $S_n$ (up to isomorphism)! \par
The idea of our proof will be to show that each $S^\lambda$ is indecomposable (contains no nontrivial submodules)
and distinct by putting a linear order on the tableaux of any shape
and of size $n$. Matchske's theorem tells us that indecomposibility is equivalent to irreducibility, and we will use some properties of the ordering
to show that each $S^\lambda$ is disjoint to $S^{\lambda'}$ when $\lambda < \lambda'$ and $\lambda \neq \lambda'$.
\par
We begin by defining the lexicographical and dominance orderings:
\begin{defn}[ {\cite[pg. 36]{fulton}} ]
  The \emph{lexicographic} ordering on partitions of size $n$, denoted $ \lambda \leq \lambda'$, means that
  for the first $i$ for which $\lambda_i \neq \lambda'_i$ (if any), has $\lambda_i < \lambda'_i$ (in the standard ordering on integers).
  It is a linear order.
\end{defn}
\begin{defn}
  The \emph{dominance} ordering on partitions of size $n$, denoted $ \lambda \normeq \lambda'$ or ``$\lambda'$ dominates $\lambda$'',
  means that $\sum_{1 \leq j \leq i} \lambda_j \leq \sum \sum_{1 \leq j \leq i} \lambda'_j$ for all $ 1 \leq i \leq \infty$. \par
  The intuition here is that parititions with a few long rows dominate partitions with many short rows; note, however, that this is
  not a linear order. For example, we see for the partitions $\lambda = (4,1,1,1) \vdash 7$ and $\lambda' = (3,3,1) \vdash 7$ that
  \begin{align*}
    &4 \not \leq 3, \text{ so $\lambda \not \normeq \lambda'$}\\
    &3 \leq  4, \ 3 + 3 \leq 4 + 1  \text{ so $\lambda' \not \normeq \lambda$}\\
  \end{align*}
  so $\lambda$ and $\lambda'$ are incomparable.
\end{defn}
We can now state the following lemma:
\begin{lemma}[ {\cite[Lemma 7.1]{fulton}} ] \label{lem1}
  Let $T$ and $T'$ be tableaux of shape $\lambda$ and $\lambda'$ respectively, each of size $n$. Note that they can have different shape!
  Assume that $\lambda$ does not strictly dominate $\lambda'$.
  Then exactly one of the following occurs:
  \begin{enumerate}
  \item There are two distinct integers that occur in the same row of $T'$ and the same column of $T$.
  \item $\lambda = \lambda'$, and there is some $p'$ in $R(T)$ and some $q$ in $C(T)$ such that $p' \cdot T' = q \cdot T$.
  \end{enumerate}
\end{lemma}
\begin{proof}
  Suppose 1 is false. The entries of the first row of $T'$ must occur in different columns of $T$,
  so there is a $q_1 \in C(T)$ so that these entires occur in the first row of $q \cdot T$.\par
  The entries of the second row of $T'$ occur in different columns in $T$, and so they also occur in different columns of $q_1 \cdot T$,
  so there is a $q_2 \in C(q_1 \cdot T) = C(T)$ that:
  \begin{enumerate}
  \item Doesn't move entries in $q_1 \cdot T$ that are also in the first row of $T'$.
  \item Moves entries in the second row of $T'$ that are in $q_1 \cdot T$
  into the second row of $q_1 \cdot T$.
  \end{enumerate}
  We can repeat this process to obtain $q_1, \dots, q_k$ such that the entries in the first $k$ rows of $T'$ occur in the first $k$ rows of
  $q_k \cdot q_{k-1} \dots \cdot q_1 \cdot T$. The actions of $q_1, \dots, q_k$ don't change the shape of $T$, so we can deduce that
  \[\sum_{1 \leq j \leq k} \lambda'_j \leq \sum_{1 \leq j \leq k} \lambda_j\]
  because each row of $\lambda$ must have at least enough boxes to contain all the entries of the corresponding row of $\lambda'$.
  This holds for all $k$, so by definition $\lambda$ dominates $\lambda'$. \par
  We assumed that $\lambda$ doesn't strictly dominate $\lambda'$, so the only possibility is that $\lambda = \lambda'$.
  We can thus take $k$ to be the number of rows in $\lambda$ and let $q = q_k \cdots q_1$ so that
  $q \cdot T$ and $T'$ have the same entries in each row.
  We can now conclude that there is some $p' \in R(T)$ such that $p' \cdot T' = q \cdot T$.
\end{proof}
We now define our needed linear ordering; note that this ordering is on all tableaux of size $n$, not just partitions!
\begin{defn}[ { \cite[pg. 84-85]{fulton} } ]
  We denote the linear ordering $T < T'$ on tableaux to mean either:
  \begin{enumerate}
  \item The shape of $T'$ is larger than the shape of $T$ in the lexicographical order.
  \item $T$ and $T'$ have the same shape, and the largest entry that is in a different box in the two numberings occurs
    earlier in the column word of $T'$ than in the column word of $T$. \\
    The column word is obtained by listing the entries of each column from bottom to top, reading columns from left to right.
  \end{enumerate}
\end{defn}
\begin{ex}
  This ordering puts the standard tableux of shape $(3,2)$ in the following order:
    \ytableausetup{notabloids,centertableaux}
  \[
    \begin{ytableau}
      1 & 2 & 3 \\
      4 & 5 \\
    \end{ytableau}
    >
    \begin{ytableau}
      1 & 2 & 4 \\
      3 & 5 \\
    \end{ytableau}
    >
    \begin{ytableau}
      1 & 3 & 4 \\
      2 & 5 \\
    \end{ytableau}
    >
    \begin{ytableau}
      1 & 2 & 5 \\
      3 & 4 \\
    \end{ytableau}
    >
    \begin{ytableau}
      1 & 3 & 5 \\
      2 & 4 \\
    \end{ytableau}.
  \]
  We demonstrate the first inequality by observing that $4$ is the largest entry in a different box between the two SYT,
  and that the first SYT has column word $41523$ while the second has column word $31524$.
  We see that $4$ occurs earlier in the column word of the first SYT, so the first SYT is ``larger'' in this ordering.
  
\end{ex}
An important property of this ordering for SYT $T$ and any $p \in R(T), q\in C(T)$ is that
\[ p \cdot T < T \text{ and } q \cdot T > T.\]
The symbol ``$<$'' does not denote a strict poset relation, so it could be that $p \cdot T = T$. \par
This is true because the largest element moved by a row permutation must be moved left, pushing it closer to the front in the column word,
while the largest element moved by a column permutation must be moved up, pushing it further back in the column word.
\begin{ex}
  In the SYT below, any nontrivial row permutation will swap at least two elements.
  Because every element to the right of a given entry in a row is larger in a SYT, this swap will move the larger element to the left.
  Analogously, elements lower in a column are larger, so a column permutation will move the larger element up.
  \[
    (15) \cdot 
    \begin{ytableau}
      1 & 2 & 3 & 4 & 5 
    \end{ytableau}
    =
    \begin{ytableau}
      5 & 2 & 3 & 4 & 1 
    \end{ytableau}
  \]
\end{ex}

We need the following results for the next 2 proofs (both being excercises from \cite{fulton}):
\begin{proof}
  We prove that, for all $q' \in C(T)$,
  \[q' \cdot b_T = \sgn(q') b_T.\]
  First we compute
  \[q' \cdot b_T = q' \cdot \sum_{q \in C(T)} \sgn(q)\{q \cdot T\} = \sum_{q \in C(T)} \sgn(q)\{q' \cdot q \cdot T\}.\]
  We know from abstract algebra that $q' C(T) = C(T)$ because $q' \in C(T) \leq S_n$; the function of applying $q'$ to every element of $C(T)$ is a
  bijection.
  Therefore the composition of $q'$ with every element of $C(T)$ will only reorder the addends in the sum.
  If $q'$ is odd, it will map each permutation to a permutation with opposite sign; if it is even, it will maintain the parity of each permutation.
  This is equivalent to multiplying  each permutation by the sign of $q'$, so
  \[\sum_{q \in C(T)} \sgn(q)\{q' \cdot q \cdot T\} = \sum_{q \in C(T)} \sgn(q)\sgn(q') \{q \cdot T\}
    = \sgn(q') \sum_{q \in C(T)} \sgn(q) \{q \cdot T\} = \sgn(q') b_T.\]
\end{proof}
\begin{proof}
  We prove that $b_T \cdot b_T = |C(T)| \cdot b_T$, where $\cdot$ here is multiplication in the group ring.
  We linearly extend the result we just proved:
  \[ b_T \cdot b_T = (\sum_{q \in C(T)} \sgn(q)q) \cdot b_T = |C(T)|\sgn(q)\sgn(q) \cdot b_T = |C(T)| \cdot b_T.\]  
\end{proof}

We now state and prove another lemma which applies our previous lemma to the $M^\lambda$ module:
\begin{lemma}[{\cite[Lemma 7.2]{fulton}}]\label{lem2}
  Let $T$ and $T'$ be numberings of shapes $\lambda$ and $\lambda'$ respectively,
  and assume that $\lambda$ does not strictly dominate $\lambda'$. \\
  \begin{enumerate}
  
  \item If there is a pair of integers in the same row of $T'$ and the same column of $T$, then $b_T \cdot \{T'\} = 0$.
  \item If there is no such pair, then $b_T \cdot \{T'\} = \pm v_T$.
  \end{enumerate}
\end{lemma}
\begin{proof}    
  If there is such a pair of integers, let $t \in S_n$ be the transposition that swaps them.
  Then $b_T \cdot t = -b_T$, since $t$ is in the column group of $T$, and transpositions have odd sign.
  On the other hand $t \cdot \{T'\} = \{T'\}$ by the definition of a tabloid, because $t$ is in the row group of $T'$.
  Therefore
  \[b_T \cdot \{T'\} = b_T \cdot (t \cdot \{T'\}) = (b_T \cdot t) \cdot \{T'\} = - b_T \cdot \{T'\}\]
  so $b_T \cdot \{T'\} = 0$.\par
  If there is no such pair of integers, then let $p'$ and $q$ be as in the second case of our first lemma.
  Then
  \begin{align*}
    b_T \cdot \{ T'\} &=b_T \{p' \cdot T'\} = b_T \cdot \{ q \cdot T\} \\
                      &= b_T \cdot q \cdot \{T\} = \sgn(q) b_T \cdot \{T\} = \sgn(q) v_T.
  \end{align*}
\end{proof}
Finally we have the tools to prove our main theorem!
\begin{thm}[{\cite[ Pg. 87-88]{fulton}}]
  For each partition $\lambda$ of $n$, $S^\lambda$ is an irreducible representation of $S_n$. Every irreducible representation of
  $S_n$ is isomorphic to exactly one $S_n$.
\end{thm}
\begin{proof}
  First, we note that no $v_T$ is $0$ by definition, so the modules $S^\lambda$ are all nonzero (nontrivial subspaces of $M^\lambda$).
  We wish to prove the following statements, for a given tableaux $T$ of $\lambda$:
  \begin{align}
  &b_T \cdot M^\lambda = b_T \cdot S^\lambda = \CC \cdot v_T \neq \{0\}. \\
  &b_T \cdot M^{\lambda'} = b_T \cdot S^{\lambda'} = \{0\} \text{ if $\lambda < \lambda'$ and $\lambda \neq \lambda'$,}
  \end{align}
  where $\{0\}$ is the trivial zero subspace.
  We begin with the first equation. \par
  The first equality follows as such, using our result that $b_T \cdot b_T = |C(T)| \cdot b_T$:
  \begin{align*}
    b_T \cdot b_T &= |C(T)| \cdot b_T \iff \\
    \dfrac{1}{|C(T)|} \cdot b_T \cdot b_T &= b_T;\\
    b_T \cdot M^\lambda &= \dfrac{1}{|C(T)|} \cdot b_T \cdot b_T \cdot M^\lambda \\
                  &=\dfrac{1}{|C(T)|} \cdot b_T S^\lambda \\
                  &= b_T \cdot \dfrac{1}{|C(T)|}  \cdot S^\lambda \\
                  &= b_T \cdot S^\lambda
  \end{align*}
  where we can commute $\dfrac{1}{|C(T)|}$ because it is only a scalar; furthermore $\dfrac{1}{|C(T)|}  \cdot S^\lambda = S^\lambda$
  because vector spaces are invariant under scaling.
  The second equality follows because we know that for any $T,T'$ tableaux of $\lambda$, we know by \cref{lem1} there are not
  two distinct integers that appear in the same row of $T'$ and in the same column of $T$, so by \cref{lem2}
  $b_T \cdot \{T'\} = \pm v_T$.
  This means that $b_t \cdot M^\lambda = \CC v_T$,
  because every element in $M^\lambda$ is mapped to a scaling of $v_T$, and any we can obtain any scaling of $v_T$ via
  $b_T \cdot x \{T\} =x v_T \in \CC \cdot v_T$. \par
  For the second equation, the first equality follows from our argument regarding the second equation.
  The second equality follows because by assumption we are in case 1 of \cref{lem1} and therefore the first case of \cref{lem2},
  so $b_T \cdot \{T'\} = 0$ for all $\{T'\} M^\lambda$, and so $b_T \cdot M^\lambda = 0$.\par
  Now by Matchske's theorem we know that $S^\lambda = W_1 \oplus W_2 \oplus \dots \oplus W_k$ for irreducible submodules $W_j$.
  We see that
  \[\CC \cdot v_T = b_T \cdot S^\lambda = b_T \cdot W_1 \oplus b_T \cdot W_2 \dots \oplus b_T \cdot W_k,\]
  so one of the modules $W_j$ must contain $v_T$.
  If some $W_i$ contains $v_T$, then by the definition of a submodule $W_i = \CC [S_n] \cdot v_T = S^\lambda$, so
  the other submodules must be the zero submodule and so $S^\lambda$ is irreducible.
  \par
  Furthermore we prove that $S^\lambda  \ncong S^{\lambda'}$ for any $\lambda < \lambda'$ and $\lambda \neq \lambda'$.
  We assume there exists a module isomorphism $\Theta$ between $S^\lambda$ and $S^{\lambda'}$.
  By the definition of a module isomorphism, $\Theta(b_T \cdot S^{\lambda'}) = b_T \cdot \Theta( S^{\lambda'})$,
  but we just showed that
  \[\Theta(b_T \cdot S^{\lambda'}) = \Theta(0) = 0 \neq b_T \cdot S^\lambda = b_T \cdot \Theta( S^{\lambda'})\]
  which is a contradiction. Therefore $S^\lambda \ncong S^{\lambda'}$, and because $<$ is a linear ordering, we can conclude that
  no two distinct $S^\lambda$ are isomorphic. \par
  Finally we cite without proof \cite[Proposition 1.10.1]{sagan}, specifically the result that the number of irreducible modules/representations equals
  the number of conjugacy classes of the group. We have previously discussed how the conjugacy classes of $S_n$ are in bijection with cycle types of $S_n$,
  which in turn are in bijection with partitions of $n$.
  Because there is one Specht module $S^\lambda$ for each partition $\lambda$ of size $n$,
  we can conclude that there are exactly as many $S^\lambda$ as there are irreducible representations of $S_n$.
  That is, the set of modules $S^\lambda$ is all of the irreducible modules/representations of $S_n$ up to isomorphism.
  \end{proof}
  
\subsection{A basis for $S^\lambda$ and further reading} \hfill\\
With the main result aside, we now prove a proposition that utilizes combinatorial results from class!
\begin{prop}[{\cite[Pg. 88]{fulton}}]\label{prop1}
  The elements $v_T$, as $T$ varies over the \emph{standard} tableaux of $\lambda$, form a basis for $S^\lambda$.
\end{prop}
\begin{proof}
  The element $v_T$ is a linear combination of $\{T\}$, with coefficent $1$ (as the trivial permutation has even sign),
  and elements $\{q \cdot T\}$, for $q \in C(T)$, with coefficents $\pm 1$.
  Recall that when $T$ is a SYT, $q \cdot T < T$, and furthermore this relation is strict when $q$ is nontrivial.
  To find solutions to the equation $\sum_{T \in \text{ SYT of $\lambda$}} x_Tv_T = 0$,
  we can look at the largest $v_T$ with nonzero coefficent $x_T$.
  We know that the $\{T\}$ component cannot be canceled out by some other $v_{T'}$, because it must be that $\{T\} \neq \{T'\}$ in
  order for the relation $v_T > v_{T'}$ to be strict,
  and furthermore $\{T\} \neq \{q \cdot T'\}$ because we know $\{ T\} > \{T'\} > \{q \cdot T'\}$ and at least the first relation is strict.
  Thus $\{T\}$ cannot be canceled out by some other term, so it must be that $x_T = 0$; but then all $x_T = 0$ because
  if any $x_T$ is nonzero there will be some largest nonzero $v_T$.
  We can thus conclude that the elements $v_T \in S^\lambda$ as $T$ varies over the SYT are linearly independent. \par
  To show that these elements span $S^\lambda$, we again cite \cite[Proposition 1.10.1]{sagan}, specifically the result
  that
  \[ \sum_i (\dim V_i)^2 =|G| \]
  where the $V_i$ are a complete set of irreducible $G$-modules.
  In our case this means that
  \[\sum_\lambda (\dim S^\lambda)^2 = n!\]
  We proved in class that
  \[ \sum_\lambda(f^\lambda)^2 = n!\]
  where $f^\lambda$ is the number of SYT of the partition $\lambda$ of size $n$.
  We can thus conclude
  \[ n! = [\sum_\lambda (\dim S^\lambda)^2 = \sum_\lambda(f^\lambda)^2 = n!.\]
  It follows that $\dim(S^\lambda) = f^\lambda$ for all $\lambda$, and because $f^\lambda$ counts the number of SYT of shape $\lambda$
  this means that the elements $v_T$ as $T$ varies over SYT must span $S^\lambda$.
\end{proof}
In this proof we used our in-class proof that $\sum_\lambda(f^\lambda)^2 = n!$, but it's actually possible to prove \cref{prop1}
without this fact using the so-called straightening algorithm \cite[Chapter 7.4]{fulton}.
Taking that route leads to a representation-theoretic proof that $\sum_\lambda(f^\lambda)^2 = n!$,
because if we know $n! = \sum_\lambda (\dim S^\lambda)^2$ and that the elements $v_T$ over SYT form a basis for $S^\lambda$,
we can conclude that $\dim(S^\lambda) = f^\lambda$ and therefore $\sum_\lambda(f^\lambda)^2 = n!$. \par
That concludes the results of this paper. The most powerful tool that representation theory provides towards results in other disciplines
actually come from a subfield called character theory, which is concerned with the trace (sum of diagonal values) of matrix representations.
Character theory is needed to prove results in abstract algebra that we have otherwise found no methods of proving --- specifically, it
is needed for the classification of finite simple groups! As such the character theory of representations of $S_n$ is of great interest
and application both inside and outside representation theory. Two other jumping off points would be the straightening algorithm
discussed in \cite[Chapter 7.4]{fulton}, which is an application of something called Garnier elements, and the ring of
representations of $S_n$ and symmetric functions \cite[7.3]{fulton}, which allows for the application of symmetric functions
to the representation theory of $S_n$ and leads to results like Youngs rule, which tells us that irreducible representations of
$S_{n+1}$ or $S_{n-1}$ can be induced or restricted from representations of $S_n$ by taking direct sums of $S^\mu$ for greater or lesser covering partitions
of $\lambda$ of $n$ in Young's diagram!






\bibliographystyle{alpha}
\bibliography{combinatoricsProject}


\end{document}


