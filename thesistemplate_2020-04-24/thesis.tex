% This is the Reed College LaTeX thesis template. Most of the work 
% for the document class was done by Sam Noble (SN), as well as this
% template. Later comments etc. by Ben Salzberg (BTS). Additional
% restructuring and APA support by Jess Youngberg (JY).
% Your comments and suggestions are more than welcome; please email
% them to cus@reed.edu
%
% See http://web.reed.edu/cis/help/latex.html for help. There are a 
% great bunch of help pages there, with notes on
% getting started, bibtex, etc. Go there and read it if you're not
% already familiar with LaTeX.
%
% Any line that starts with a percent symbol is a comment. 
% They won't show up in the document, and are useful for notes 
% to yourself and explaining commands. 
% Commenting also removes a line from the document; 
% very handy for troubleshooting problems. -BTS

% As far as I know, this follows the requirements laid out in 
% the 2002-2003 Senior Handbook. Ask a librarian to check the 
% document before binding. -SN

%%
%% Preamble
%%
% \documentclass{<something>} must begin each LaTeX document
\documentclass[12pt,twoside]{reedthesis}
% Packages are extensions to the basic LaTeX functions. Whatever you
% want to typeset, there is probably a package out there for it.
% Chemistry (chemtex), screenplays, you name it.
% Check out CTAN to see: http://www.ctan.org/
%%
\usepackage{graphicx,latexsym} 
\usepackage{amssymb,amsthm,amsmath}
\usepackage{longtable,booktabs,setspace}
\usepackage{braket}

\usepackage[hyphens]{url}
\usepackage{rotating}
%\usepackage{natbib}
\usepackage[numbers]{natbib}
% Comment out the natbib line above and uncomment the following two lines to use the new 
% biblatex-chicago style, for Chicago A. Also make some changes at the end where the 
% bibliography is included. 
%\usepackage{biblatex-chicago}
%\bibliography{thesis}

% \usepackage{times} % other fonts are available like times, bookman, charter, palatino

% beware, math preamble paste begins

\theoremstyle{plain}   % This is the default, anyway
\newtheorem{thm}{Theorem}[section] % numbered theorem
\makeatletter\let\c@thm\c@thm\makeatother
\newtheorem{cor}{Corollary}[section]
\makeatletter\let\c@cor\c@thm\makeatother
\newtheorem{lemma}{Lemma}[section]
\makeatletter\let\c@lemma\c@thm\makeatother
\newtheorem{prop}{Proposition}[section]
\makeatletter\let\c@prop\c@thm\makeatother
\newtheorem{claim}{Claim}[section]
\makeatletter\let\c@claim\c@thm\makeatother


\newtheorem*{unnumberedtheorem}{Theorem}  % unnumbered theorem
\newtheorem*{unnumberedcorollary}{Corollary}
\newtheorem*{unnumberedlemma}{Lemma}
\newtheorem*{unnumberedproposition}{Proposition}
\newtheorem*{unnumberedclaim}{Claim}

\theoremstyle{definition}

\newtheorem{defn}{Definition}[section]
\makeatletter\let\c@defn\c@thm\makeatother
\newtheorem{const}{Construction}[section]
\makeatletter\let\c@const\c@thm\makeatother
\newtheorem{notn}{Notation}[section]
\makeatletter\let\c@notn\c@thm\makeatother
\newtheorem{outline}{Proof Outline}[section]
\makeatletter\let\c@outline\c@thm\makeatother
\newtheorem{post}{Postulate}[section]
\makeatletter\let\c@outline\c@thm\makeatother


\theoremstyle{remark}

\newtheorem{rem}{Remark}[section]
\makeatletter\let\c@rem\c@thm\makeatother
\newtheorem{ex}{Example}[section]
\makeatletter\let\c@ex\c@thm\makeatother
\newtheorem{observation}{Observation}[section]
\makeatletter\let\c@observationn\c@thm\makeatother

\newtheorem*{unnumberedtheoremremark}{Remark}
\newtheorem*{unnumberedtheoremexample}{Example}
\newtheorem*{unnumberedtheoremdefinition}{Defintion}

\makeatletter
\let\c@equation\c@thm
\numberwithin{equation}{section}
\makeatother

\def\cA{\mathcal{A}}\def\cB{\mathcal{B}}\def\cC{\mathcal{C}}\def\cD{\mathcal{D}}\def\cE{\mathcal{E}}\def\cF{\mathcal{F}}\def\cG{\mathcal{G}}\def\cH{\mathcal{H}}\def\cI{\mathcal{I}}\def\cJ{\mathcal{J}}\def\cK{\mathcal{K}}\def\cL{\mathcal{L}}\def\cM{\mathcal{M}}\def\cN{\mathcal{N}}\def\cO{\mathcal{O}}\def\cP{\mathcal{P}}\def\cQ{\mathcal{Q}}\def\cR{\mathcal{R}}\def\cS{\mathcal{S}}\def\cT{\mathcal{T}}\def\cU{\mathcal{U}}\def\cV{\mathcal{V}}\def\cW{\mathcal{W}}\def\cX{\mathcal{X}}\def\cY{\mathcal{Y}}\def\cZ{\mathcal{Z}}

\def\AA{\mathbb{A}} \def\BB{\mathbb{B}} \def\CC{\mathbb{C}} \def\DD{\mathbb{D}} \def\EE{\mathbb{E}} \def\FF{\mathbb{F}} \def\GG{\mathbb{G}} \def\HH{\mathbb{H}} \def\II{\mathbb{I}} \def\JJ{\mathbb{J}} \def\KK{\mathbb{K}} \def\LL{\mathbb{L}} \def\MM{\mathbb{M}} \def\NN{\mathbb{N}} \def\OO{\mathbb{O}} \def\PP{\mathbb{P}} \def\QQ{\mathbb{Q}} \def\RR{\mathbb{R}} \def\SS{\mathbb{S}} \def\TT{\mathbb{T}} \def\UU{\mathbb{U}} \def\VV{\mathbb{V}} \def\WW{\mathbb{W}} \def\XX{\mathbb{X}} \def\YY{\mathbb{Y}} \def\ZZ{\mathbb{Z}}  

\def\fa{\mathfrak{a}} \def\fb{\mathfrak{b}} \def\fc{\mathfrak{c}} \def\fd{\mathfrak{d}} \def\fe{\mathfrak{e}} \def\ff{\mathfrak{f}} \def\fg{\mathfrak{g}} \def\fh{\mathfrak{h}} \def\fj{\mathfrak{j}} \def\fk{\mathfrak{k}} \def\fl{\mathfrak{l}} \def\fm{\mathfrak{m}} \def\fn{\mathfrak{n}} \def\fo{\mathfrak{o}} \def\fp{\mathfrak{p}} \def\fq{\mathfrak{q}} \def\fr{\mathfrak{r}} \def\fs{\mathfrak{s}} \def\ft{\mathfrak{t}} \def\fu{\mathfrak{u}} \def\fv{\mathfrak{v}} \def\fw{\mathfrak{w}} \def\fx{\mathfrak{x}} \def\fy{\mathfrak{y}} \def\fz{\mathfrak{z}}
\def\fN{\mathfrak{N}}
\def\fgl{\mathfrak{gl}}  \def\fsl{\mathfrak{sl}}  \def\fso{\mathfrak{so}}  \def\fsp{\mathfrak{sp}}  
\def\GL{\mathrm{GL}} \def\SL{\mathrm{SL}}  \def\SP{\mathrm{SL}}\def\OG{\mathrm{O}}

\def\aa{\mathbf{a}} \def\bb{\mathbf{b}} \def\cc{\mathbf{c}} \def\dd{\mathbf{d}} \def\ee{\mathbf{e}} \def\ff{\mathbf{f}}
%\def\gg{\mathbf{g}}
\def\hh{\mathbf{h}} \def\ii{\mathbf{i}} \def\jj{\mathbf{j}} \def\kk{\mathbf{k}}
%\def\ll{\mathbf{l}}
\def\mm{\mathbf{m}} \def\nn{\mathbf{n}} \def\oo{\mathbf{o}} \def\pp{\mathbf{p}} \def\qq{\mathbf{q}} \def\rr{\mathbf{r}} \def\ss{\mathbf{s}}  \def\uu{\mathbf{u}} \def\vv{\mathbf{v}} \def\ww{\mathbf{w}} \def\xx{\mathbf{x}} \def\yy{\mathbf{y}} \def\zz{\mathbf{z}}
\def\zzero{\mathbf{0}}

\def\<{\langle} \def\>{\rangle}
\def\Aut{\mathrm{Aut}}
\def\ch{ \stackrel}
\def\col{\mathrm{col}}
\def\dim{\mathrm{dim}} 
\def\End{\mathrm{End}} 
\def\ev{\mathrm{ev}} 
\def\f{\varphi}
\def\gcd{\mathrm{gcd}}
\def\half{\hbox{$\frac12$}}
\def\Hom{\mathrm{Hom}}
\def\img{\mathrm{img}}
\def\id{\mathrm{id}} 
\def\Inn{\mathrm{Inn}}
\def\lcm{\mathrm{lcm}}
\def\normeq{\trianglelefteq}
%\def\ch{ \stackrel{\mathrm{ch}}{\trianglelefteq}}
%\def\normeq{\trianglelefteq}
\def\nul{\mathrm{nullity}}
\def\row{\mathrm{row}}
\def\rk{\mathrm{rank}}
\def\sgn{\mathrm{sgn}}
\def\sp{\mathrm{span}}
\def\supp{\mathrm{supp}}
\def\Syl{\mathrm{Syl}}
\def\tr{\mathrm{tr}} 
\def\vep{\varepsilon}

%\usepackage{mathabx}
\def\acts{\circlearrowright} %group action

 %Cleveref definitions
% \crefname{lemma}{Lemma}{Lemmas}
% \crefname{thm}{Theorem}{Theorems}
% \crefname{defn}{Definition}{Definitions}
% \crefname{notn}{Notation}{Notations}
% \crefname{const}{Construction}{Constructions}
% \crefname{prop}{Proposition}{Propositions}
% \crefname{rem}{Remark}{Remarks}
% \crefname{cor}{Corollary}{Corollaries}
% \crefname{equation}{Diagram}{Diagrams}
% \crefname{ex}{Example}{Examples}

% Arrays:
\newcommand\Pmatrix[1]{\begin{pmatrix}#1\end{pmatrix}}
\newcommand\smatrix[1]{\text{\small$\begin{pmatrix}#1\end{pmatrix}$}}
\newcommand\fmatrix[1]{\text{\footnotesize$\begin{pmatrix}#1\end{pmatrix}$}}
\newcommand\tmatrix[1]{\text{\tiny$\begin{pmatrix}#1\end{pmatrix}$}}


% math preamble ends





\title{Representation Theory in Quantum Computing}
\author{Dale D. Schandelmeier-Lynch}
% The month and year that you submit your FINAL draft TO THE LIBRARY (May or December)
\date{May 2025}
\division{Mathematical and Natural Sciences}
\advisor{Jamie Pommershiem}
%If you have two advisors for some reason, you can use the following
\altadvisor{Greg Anderson}
%%% Remember to use the correct department!
\department{Mathematics and Computer Science}
% if you're writing a thesis in an interdisciplinary major,
% uncomment the line below and change the text as appropriate.
% check the Senior Handbook if unsure.
\thedivisionof{The Established Interdisciplinary Committee for}
% if you want the approval page to say "Approved for the Committee",
% uncomment the next line
\approvedforthe{Committee}

\setlength{\parskip}{0pt}
%%
%% End Preamble
%%
%% The fun begins:
\begin{document}

  \maketitle
  \frontmatter % this stuff will be roman-numbered
  \pagestyle{empty} % this removes page numbers from the frontmatter

% Acknowledgements (Acceptable American spelling) are optional
% So are Acknowledgments (proper English spelling)
    \chapter*{Acknowledgements}
	I want to thank a few people.

% The preface is optional
% To remove it, comment it out or delete it.
    \chapter*{Preface}
	This is an example of a thesis setup to use the reed thesis document class.
	
	

    \chapter*{List of Abbreviations}
		You can always change the way your abbreviations are formatted. Play around with it yourself, use tables, or come to CUS if you'd like to change the way it looks. You can also completely remove this chapter if you have no need for a list of abbreviations. Here is an example of what this could look like:

	\begin{table}[h]
	\centering % You could remove this to move table to the left
	\begin{tabular}{ll}
		\textbf{ABC}  	&  American Broadcasting Company \\
		\textbf{CBS}  	&  Columbia Broadcasting System\\
		\textbf{CDC}  	&  Center for Disease Control \\
		\textbf{CIA}  	&  Central Intelligence Agency\\
		\textbf{CLBR} 	&  Center for Life Beyond Reed\\
		\textbf{CUS}  	&  Computer User Services\\
		\textbf{FBI}  	&  Federal Bureau of Investigation\\
		\textbf{NBC}  	&  National Broadcasting Corporation\\
	\end{tabular}
	\end{table}
	

    \tableofcontents
% if you want a list of tables, optional
    \listoftables
% if you want a list of figures, also optional
    \listoffigures

% The abstract is not required if you're writing a creative thesis (but aren't they all?)
% If your abstract is longer than a page, there may be a formatting issue.
    \chapter*{Abstract}
	The preface pretty much says it all.
	\onehalfspacing
% \doublespacing

	The introduction, yay!


    \chapter{Introduction}

    \chapter{Background}
    
   We first introduce quantum computing and representation theory.
   \section{Quantum Computing}
   Quantum computing exploits the nonclassical nature of quantum mechanics to achieve asymptotic improvements in runtime over classical algorithms.
   Not all problems admit a quantum speedup; we will be more specific about our specific computational framework after the neccesary background.
   To illuminate the physical-mathematical framework in which quantum computing takes place, we will expound some neccesary linear algebra and the postulates of quantum mechanics from \cite{neilsen2010}[Chapter 2].
   \subsection{Quantum Mechanics}
   Throughout this section we use the ``bra-ket'' notation to represents vectors in vector spaces. This is inherited from physics and uses the symbols
   $ket{\cdot}$ called ket, $bra{\cdot}$ called bra, and $braket{\cdot | \cdot}$ called braket (as we have ``bracketed'' the symbols); 
   \[ \ket{\psi} \]
   is a vector we have labeled $\psi$. In quantum computing we are concerned with complex vector spaces by postulate; we need to define inner products both for this postulate and to elaborate on bra-ket notation.
   \begin{defn}[Inner product, \cite{neilsen2010}[2.1.4]]
     A function $(\cdot, \cdot): V \times V \to \CC$ is a (complex) inner product if it satisfies the following requirements:
     \begin{enumerate}
     \item $(\cdot, \cdot)$ is linear in the second argument:
       \[ \left( \ket{v}, \sum_i \lambda_i \ket{w_i} \right) = \sum_i \lambda_i \left( \ket{v}, \ket{w_i} \right).\]
     \item $(\ket{v}, \ket{w}) = (\ket{v}, \ket{w})^*$.
     \item $(\ket{v}, \ket{v}) \geq 0$ with equality if and and only if $\ket{v} = 0$.
     \end{enumerate}
   \end{defn}
   The intuition of braket notation is that $\bra{\phi}$ is the dual of the vector $\ket{\phi}$, which we notate in matrices turning a column vector into a row vector. Therefore taking the standard matrix product
   \[\bra{\phi} \ket{\psi}\]
   is equivalent to the standard inner product $(\ket{\phi}, \ket{\psi}) = \sum_{k=1}^n \phi_k^* \psi_k$, which we notate as
   \[\braket{\phi | \psi};\]
   here the symbol $\cdot^*$ represents the unary complex conjugation operation.
   The elegance of braket notation, apart from saving space, is that we can easily tell the output of complicated linear expressions; it easy to see that $\braket{\phi | \psi} \braket{x | y}$ outputs a scalar, while it takes a little thinking to see that
   \[
     \begin{bmatrix}
       x & y
     \end{bmatrix}
     \begin{bmatrix}
       x \\ y
     \end{bmatrix}
     \begin{bmatrix}
       x & y
     \end{bmatrix}
     \begin{bmatrix}
       x \\ y
     \end{bmatrix}
   \]
   outputs a scalar. \par
   We are now ready to introduce our first postulate. As mathematicians we needn't worry about how pysicists experimentally constructed these postulates, although their
   (relative) truth is neccesary if we are to physically realize quantum computers and employ the alogrithms we design.
   If we take these postulates at face value, phenomenon that are strange from a classical physics standpoint feel like natural consequences of the mathematics;
   in that sense it can be helpful to abstract away your physical intution and instead focus on the consequences of these postulates when we start doing computation.
   \begin{post}[\cite{neilsen2010}[2.2.1]]
     Associated to any isolated physical system is a complex vector space
     with inner product (that is, a Hilbert space) known as the \emph{state space} of the
     system. The system is completely described by its \emph{state vector}, which is a unit
     vector in the system's state space.
   \end{post}
   In our quantum computing model we regard these physical systems as memory and perform operations on this memory through the appropriate mathematical morphism, which in this case are linear operators.
   A minor detail is that Hilbert spaces have more properties when they are infinite dimensional, but in our case we will only ever use finite memory and so our hilbert spaces are always finite dimensional.
   The simplest quantum mechanical system is the \emph{qubit}. A qubit resides in $\CC^2$ and so has a two dimensional state space.
   We use $\ket{0}$ and $\ket{1}$ to label the elements of an orthonormal basis for the qubit; in practice we let
   \[\ket{0} =
     \begin{pmatrix}
       1 \\
       0
     \end{pmatrix},
     \quad
     \ket{1} =
     \begin{pmatrix}
       0 \\
       1
     \end{pmatrix}.
   \]
   Note that $\ket{0}$ is not the zero vector $0$; we use $\ket{0}$ despite this overlap in reference to the two states of a bit, $0$ and $1$.
   What makes the qubit interesting in contrast is that we are not limited to state vectors $\ket{0}$ and $\ket{1}$; we know from linear algebrea that
   we can express any element of the state basis as
   \[\ket{\phi} = a \ket{0} + b \ket{1}, \]
   where $a$ and $b$ are in the complex numbers. Even with the condition that $\ket{\phi}$ be a unit vector --- that is, that $\braket{\phi | \phi} = 1$ ---
   the state of the qubit is described by 4 real numbers instead of one bit! In practice
   \chapter{Results}

















        









































































        
    \chapter{The First}
    	This is the first page of the first chapter. You may delete the contents of this chapter so you can add your own text; it's just here to show you some examples. 
	
\section{References, Labels, Custom Commands and Footnotes}
It is easy to refer to anything within your document using the \texttt{label} and \texttt{ref} tags.  Labels must be unique and shouldn't use any odd characters; generally sticking to letters and numbers (no spaces) should be fine. Put the label on whatever you want to refer to, and put the reference where you want the reference. \LaTeX\ will keep track of the chapter, section, and figure or table numbers for you. 

\subsection{References and Labels}
Sometimes you'd like to refer to a table or figure, e.g. you can see in Figure \ref{subd2} that you can rotate figures . Start by labeling your figure or table with the label command (\verb=\label{labelvariable}=) below the caption (see the chapter on graphics and tables for examples). Then when you would like to refer to the table or figure, use the ref command (\verb=\ref{labelvariable}=). Make sure your label variables are unique; you can't have two elements named ``default." Also, since the reference command only puts the figure or table number, you will have to put  ``Table" or ``Figure" as appropriate, as seen in the following examples:

 As I showed in Table \ref{inheritance} many factors can be assumed to follow from inheritance. Also see the Figure \ref{subd} for an illustration.
 
\subsection{Custom Commands}\label{commands}
Are you sick of writing the same complex equation or phrase over and over? 

The custom commands should be placed in the preamble, or at least prior to the first usage of the command. The structure of the \verb=\newcommand= consists of the name of the new command in curly braces, the number of arguments to be made in square brackets and then, inside a new set of curly braces, the command(s) that make up the new command. The whole thing is sandwiched inside a larger set of curly braces. 

% Note: you cannot use numbers in your commands!
\newcommand{\hydro}{H$_2$SO$_4$}

In other words, if you want to make a shorthand for H$_2$SO$_4$, which doesn't include an argument, you would write: \verb=\newcommand{\hydro}{H$_2$SO$_4$}= and then when you needed  to use the command you would type \verb=\hydro=. (sans verb and the equals sign brackets, if you're looking at the .tex version). For example: \hydro

\subsection{Footnotes and Endnotes}
	You might want to footnote something.\footnote{footnote text} Be sure to leave no spaces between the word immediately preceding the footnote command and the command itself. The footnote will be in a smaller font and placed appropriately. Endnotes work in much the same way. More information can be found about both on the CUS site.
	
\section{Bibliographies}
	Of course you will need to cite things, and you will probably accumulate an armful of sources. This is why BibTeX was created. For more information about BibTeX and bibliographies, see our CUS site (\url{web.reed.edu/cis/help/latex/index.html}). There are three pages on this topic: {\it bibtex} (which talks about using BibTeX, at \url{/latex/bibtex.html}), {\it bibtexstyles} (about how to find and use the bibliography style that best suits your needs, at \url{/latex/bibtexstyles.html}) and {\it bibman} (which covers how to make and maintain a bibliography by hand, without BibTeX, at at \url{/latex/bibman.html}). The last page will not be useful unless you have only a few sources. There used to be APA stuff here, but we don't need it since I've fixed this with my apa-good natbib style file.
	
\subsection{Tips for Bibliographies}
\begin{enumerate}
\item Like with thesis formatting, the sooner you start compiling your bibliography for something as large as thesis, the better. Typing in source after source is mind-numbing enough; do you really want to do it for hours on end in late April? Think of it as procrastination.
\item The cite key (a citation's label) needs to be unique from the other entries.
\item When you have more than one author or editor, you need to separate each author's name by the word ``and'' e.g.\\ \verb+Author = {Noble, Sam and Youngberg, Jessica},+.
\item Bibliographies made using BibTeX (whether manually or using a manager) accept LaTeX markup, so you can italicize and add symbols as necessary.
\item To force capitalization in an article title or where all lowercase is generally used, bracket the capital letter in curly braces.
\item You can add a Reed Thesis citation option. The best way to do this is to use the phdthesis type of citation, and use the optional ``type'' field to enter ``Reed thesis'' or ``Undergraduate thesis''. Here's a test of Chicago, showing the second cite in a row being different. Also the second time not in a row should be different. Of course in other styles they'll all look the same.
\end{enumerate}
\section{Anything else?}
If you'd like to see examples of other things in this template, please contact CUS (email cus@reed.edu) with your suggestions. We love to see people using \LaTeX\ for their theses, and are happy to help.


\chapter{Mathematics and Science}	
\section{Math}
	\TeX\ is the best way to typeset mathematics. Donald Knuth designed \TeX\ when he got frustrated at how long it was taking the typesetters to finish his book, which contained a lot of mathematics. 
	
	If you are doing a thesis that will involve lots of math, you will want to read the following section which has been commented out. If you're not going to use math, skip over this next big red section. (It's red in the .tex file but does not show up in the .pdf.)

\chapter{Tables and Graphics}

\section{Tables}
	The following section contains examples of tables, most of which have been commented out for brevity. (They will show up in the .tex document in red, but not at all in the .pdf). For more help in constructing a table (or anything else in this document), please see the LaTeX pages on the CUS site. 

\begin{table}[htbp] % begins the table floating environment. This enables LaTeX to fit the table where it works best and lets you add a caption.
\caption[Correlation of Inheritance Factors between Parents and Child]{Correlation of Inheritance Factors between Parents and Child} 
% The words in square brackets of the caption command end up in the Table of Tables. The words in curly braces are the caption directly over the table.
\begin{center} 
% makes the table centered
\begin{tabular}{c c c c} 
% the tabular environment is used to make the table itself. The {c c c c} specify that the table will have four columns and they will all be center-aligned. You can make the cell contents left aligned by replacing the Cs with Ls or right aligned by using Rs instead. Add more letters for more columns, and pipes (the vertical line above the backslash) for vertical lines. Another useful type of column is the p{width} column, which forces text to wrap within whatever width you specify e.g. p{1in}. Text will wrap badly in narrow columns though, so beware.
\toprule % a horizontal line, slightly thicker than \hline, depends on the booktabs package
  Factors &  Correlation between Parents \& Child & Inherited \\ % the first row of the table. Separate columns with ampersands and end the line with two backslashes. An environment begun in one cell will not carry over to adjacent rows.
  \midrule % another horizontal line
	Education 				& -0.49 & Yes 	 \\ % another row
	Socio-Economic Status 	& 0.28 	& Slight \\
	Income 					& 0.08 	& No	 \\
	Family Size 			& 0.19 	& Slight \\
	Occupational Prestige 	& 0.21 	& Slight \\
\bottomrule % yet another horizontal line
\end{tabular}
\end{center}
\label{inheritance} % labels are useful when you have more than one table or figure in your document. See our online documentation for more on this.
\end{table}

	\clearpage 
%% \clearpage ends the page, and also dumps out all floats. 
%% Floats are things like tables and figures.

If you want to make a table that is longer than a page, you will want to use the longtable environment. Uncomment the table below to see an example, or see our online documentation.

   
   \section{Figures}
   
	If your thesis has a lot of figures, \LaTeX\ might behave better for you than that other word processor.  One thing that may be annoying is the way it handles ``floats'' like tables and figures. \LaTeX\ will try to find the best place to put your object based on the text around it and until you're really, truly done writing you should just leave it where it lies.   There are some optional arguments to the figure and table environments to specify where you want it to appear; see the comments in the first figure.

	If you need a graphic or tabular material to be part of the text, you can just put it inline. If you need it to appear in the list of figures or tables, it should be placed in the floating environment. 
	
	To get a figure from StatView, JMP, SPSS or other statistics program into a figure, you can print to pdf or save the image as a jpg or png. Precisely how you will do this depends on the program: you may need to copy-paste figures into Photoshop or other graphic program, then save in the appropriate format.
	
	Below we have put a few examples of figures. For more help using graphics and the float environment, see our online documentation.
	
	And this is how you add a figure with a graphic:
	\begin{figure}[h]
	% the options are h = here, t = top, b = bottom, p = page of figures.
	% you can add an exclamation mark to make it try harder, and multiple
	% options if you have an order of preference, e.g.
	% \begin{figure}[h!tbp]
	   
	       \centering
	    % DO NOT ADD A FILENAME EXTENSION TO THE GRAPHIC FILE
	    \includegraphics{subdivision}
	     \caption{A Figure}
	 \label{subd}
	\end{figure}

\clearpage %% starts a new page and stops trying to place floats such as tables and figures

\section{More Figure Stuff}
You can also scale and rotate figures.
 	\begin{figure}[h!]
	   
	       \centering
	    % DO NOT ADD A FILENAME EXTENSION TO THE GRAPHIC FILE
	    \includegraphics[scale=0.5,angle=180]{subdivision}
	    % if your figure shows up not where you want it, it may just be too big to fit. You can use the scale argument to shrink it, e.g. scale=0.85 is 85 percent of the original size. 
	     \caption{A Smaller Figure, Flipped Upside Down}
	 \label{subd2}
	\end{figure}

\section{Even More Figure Stuff}
With some clever work you can crop a figure, which is handy if (for instance) your EPS or PDF is a little graphic on a whole sheet of paper. The viewport arguments are the lower-left and upper-right coordinates for the area you want to crop.

 	\begin{figure}[h!]
	    	       \centering
	    % DO NOT ADD A FILENAME EXTENSION TO THE GRAPHIC FILE
	   \includegraphics[clip=true, viewport=.0in .0in 1in 1in]{subdivision}
	    \caption{A Cropped Figure}
	 \label{subd3}
	\end{figure}
	
      \subsection{Common Modifications}
      The following figure features the more popular changes thesis students want to their figures. This information is also on the web at \url{web.reed.edu/cis/help/latex/graphics.html}.
    %\renewcommand{\thefigure}{0.\arabic{figure}} 	% Renumbers the figure to the type 0.x
    %\addtocounter{figure}{4} 						% starts the figure numbering at 4
    \begin{figure}[htbp]
    \begin{center}
   \includegraphics[scale=0.5]{subdivision}
    \caption[Subdivision of arc segments]{\footnotesize{Subdivision of arc segments. You can see that $ p_3 = p_6^\prime$.}} %the special ToC caption is in square brackets. The \footnotesize makes the figure caption smaller
    \label{barplot}
    \end{center}
    \end{figure} 

\chapter*{Conclusion}
         \addcontentsline{toc}{chapter}{Conclusion}
	\chaptermark{Conclusion}
	\markboth{Conclusion}{Conclusion}
	\setcounter{chapter}{4}
	\setcounter{section}{0}
	
Here's a conclusion, demonstrating the use of all that manual incrementing and table of contents adding that has to happen if you use the starred form of the chapter command. The deal is, the chapter command in \LaTeX\ does a lot of things: it increments the chapter counter, it resets the section counter to zero, it puts the name of the chapter into the table of contents and the running headers, and probably some other stuff. 

So, if you remove all that stuff because you don't like it to say ``Chapter 4: Conclusion'', then you have to manually add all the things \LaTeX\ would normally do for you. Maybe someday we'll write a new chapter macro that doesn't add ``Chapter X'' to the beginning of every chapter title.

\section{More info}
And here's some other random info: the first paragraph after a chapter title or section head \emph{shouldn't be} indented, because indents are to tell the reader that you're starting a new paragraph. Since that's obvious after a chapter or section title, proper typesetting doesn't add an indent there. 


%If you feel it necessary to include an appendix, it goes here.
    % \appendix
    %   \chapter{The First Appendix}
    %   \chapter{The Second Appendix, for Fun}


%This is where endnotes are supposed to go, if you have them.
%I have no idea how endnotes work with LaTeX.

  \backmatter % backmatter makes the index and bibliography appear properly in the t.o.c...

% if you're using bibtex, the next line forces every entry in the bibtex file to be included
% in your bibliography, regardless of whether or not you've cited it in the thesis.
    \nocite{*}

% Rename my bibliography to be called "Works Cited" and not "References" or ``Bibliography''
% \renewcommand{\bibname}{Works Cited}

%    \bibliographystyle{bsts/mla-good} % there are a variety of styles available; 
%  \bibliographystyle{plainnat}
% replace ``plainnat'' with the style of choice. You can refer to files in the bsts or APA 
% subfolder, e.g. 
 \bibliographystyle{annotate}  % or
 \bibliography{thesisdsldesktop}
 % Comment the above two lines and uncomment the next line to use biblatex-chicago.
 %\printbibliography[heading=bibintoc]

% Finally, an index would go here... but it is also optional.
\end{document}
